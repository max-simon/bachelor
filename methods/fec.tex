In order to understand state transitions in physical systems, the free energy is a key quantity. It is directly linked to the probability distribution for different states and other quantities can be derived easily.\\
The free energy depends on the partition function and therefore on the considered thermodynamic ensemble. In this section, only the Helmholtz free energy $A$, which refers to a canonical ensemble, is examined, but the concepts can be applied to other ensembles as well.
\subsubsection{Partition function}
In the microcanonical ensemble, the probability that a system enters a microstate with energy $E' = E(\mathbf{q}, \mathbf{p})$ ($\mathbf{q}, \mathbf{p}$ are the positions and momenta of the particles respectively), is equal for all $\left|E' - E\right| < \diff E$ and $0$ otherwise. Therefore, the partition function $\Omega$ is given as
\begin{equation}
\label{eq:partfunc}
\Omega(N, V, E) = C_0 \int \delta\left(\mathcal{H}(\mathbf{q}, \mathbf{p}) - E\right) \diff \mathbf{q} \diff \mathbf{p},
\end{equation}
where $\mathcal{H}(\mathbf{q}, \mathbf{p}) = U(\mathbf{q}) + K(\mathbf{p})$ is the Hamiltonian and $C_0$ a proportionality constant, in which the smallest phase space volume and the indistinguishability of particles have to be taken into account \autocite[16]{freeEnergyBook}.\\
\\
In the canonical ensemble, however, the temperature $T$ is kept constant instead of the energy. Therefore, the partition function has to include all possible energies weighted with their probability given by the Boltzmann factor:
\begin{align}
Q(N, V, T) &= \int \exp\left(-\beta E\right) \Omega(N, V, E) \diff E\\
&= C_0 \int \exp\left(-{\beta\mathcal{H}(\mathbf{q}, \mathbf{p})}\right) \diff \mathbf{q} \diff \mathbf{p},
\end{align}
where $\beta = \frac{1}{k_B T}$.
The configurational integral $Z$ is defined as 
\begin{equation}
Z(N, V, T) = \int \exp\left(-{\beta U}(\mathbf{q})\right) \diff \mathbf{q}.
\end{equation}
It is important to see that $\mathcal{H}$ only depends $\mathbf{p}^2$, so the integral over the momenta can always be solved analytically by turning it into a Gaussian integral. This implies that for two related systems in which the particle masses are the same, the integral over $\mathbf{p}$ does not change, and therefore
\begin{equation}
\frac{Q_2}{Q_1} = \frac{Z_2}{Z_1}
\end{equation}
holds \autocite[17]{freeEnergyBook}.\\
\\
The Helmholtz free energy $A$ is defined as
\begin{equation}
	A = - \beta \ln\left(Q\right).
\end{equation}
Usually, the exact value of the free energy is unknown, but the main interest lies in free energy differences between two states of a system. The difference $\Delta A$ can be written as
\begin{equation}
\Delta A = A_2 - A_1 = - \beta \ln\left(Q_2\right) + \beta \ln\left(Q_1\right) = -\beta \ln\left(\frac{Q2}{Q1}\right) = -\beta \ln \left(\frac{Z_2}{Z_1}\right),
\end{equation}
and the normalisation constant $C_0$ in \autoref{eq:partfunc} therefore cancels out.
\subsubsection{Free energy in MD and Umbrella Sampling}
The free energy of a system as a function of a parameter $\xi$ is given as
\begin{equation}
\label{eq:free_energy_from_rho}
A(\xi) = - \beta \ln\left(\rho(\xi)\right).
\end{equation}
$\rho(\xi)$ is the probability distribution of $\xi$ and can be easily obtained in MD simulation by counting the number of the appropriate visited states. Therefore, it is also referred to as a histogram. $\xi$ is called reaction coordinate and could be e.g. the distance between two molecules.\\
However, in reality $A(\xi)$ can have big barriers, and because the potential energy $U$ is sharply distributed around its mean in $NVT$ simulations, $\rho(\xi)$ can hardly be sampled at a realistic computational cost.\\
\\
One way to overcome this sampling problem is umbrella sampling \autocite{originalUmbrellaSampling}. In this approach, the path $\xi$ is split up into distinct windows $[\xi_0, \xi_n]$. To each window $i$, a biasing potential $\hat{U}_i(\xi)$ can be applied to ensure a sufficient sampling of $\xi$ around $\xi_i$. This changes the potential energy to
\begin{equation}
U_{B, i}(\mathbf{q}) = U(\mathbf{q}) + \hat{U}_i(\xi(\mathbf{q})))
\end{equation}
After a simulation with the biased potential $U_{B, i}$ the unbiased measured probability distribution $\tilde{\rho}_i(\xi)$ has to be reconstructed from the observed biased one, $\tilde{\rho}_{B, i}(\xi)$. At this point, only the result is given; the full derivation can be found in \textcite[p. 179ff]{UnderstandingMD}.\\
 \begin{equation}
 \label{eq:reconstruction_from_biased}
 \tilde{\rho}_{i}(\xi) = \exp\left(-\beta\left(\Delta A_i - \hat{U}_i(\xi)\right)\right)\tilde{\rho}_{B, i}(\xi)
 \end{equation}
Of course $\Delta A_i$ is not known, but assuming that $\rho(\xi)$ is a continuous function, the results from the single windows can be combined and afterwards normalized.\\
With this method, however, only two histograms can be considered in the overlapping region. Another problem is that the sampling in the tails is usually poor and statistical errors, which propagate through all overlapping regions, can become very large \autocite[236ff]{freeEnergyBook}. Therefore, umbrella sampling is usually combined with the weighted histogram analysis method (WHAM) \autocites{originalWHAM, extensionWHAM}. This algorithm is able to combine several histograms in one overlapping region and is designed to keep the statistical errors small. The main idea is to combine probability densities linearly with an additional weighting factor $\omega_i(\xi)$ to the total probability density $\tilde{\rho}$. The weighting factors $\omega_i$ are chosen iteratively in a way that the overall statistical error is minimized.\\