\chapter{Conclusion}
The aim of this work was to gain insights into multiple FAK interactions and their impacts on conformational changes using MD. Because these systems are very large the coarse graining force field \martini{} was used.\\
\\
In order to avoid falling of FAK on the membrane we introduced a stabilizing force, which acts on the FERM domain of the FAK molecules. In \autoref{forceana} we showed that this approach doesn't influence the observables used in the remaining part. However, it has important limitations. The force does not only prevent tiltings around the long axis of FAK (falling to side), but also around the short axis, which happens f.e. in FERM-FERM dimerization. Another problem is the absolute reference to the z axis. A reference to the membrane would be better to allow tiltings arising from membrane curvature.\\ % further studies?
\\
From the configurations of FAK in solution we identified important residues contributing to the FERM kinase interface. The observations fit well with experimental studies of \textcite{structFAK}. Also the burying of the active site of the kinase as well as the hiding of the autophosphorylation site was observed in these simulations.\\
We recognized configurational changes of FAK when it binds to a \pip{} containing membrane, namely a partial opening, the promotion of the autophosphorylation site and that the domains stays associated. These changes are consistent with previous studies on the influence of \pip{} binding to FAK by \textcite{pap001} and \textcite{pap003}. Also the free energy profile of the \pip{} binding site in the FERM domain we obtained from \martini{} simulations samples th equivalent profile from \charmm{} simulations in the range of the statistical error.\\
Albeit \martini{} is a coarse graining force field, even atomistic details were reproduced, which also show the power of \martini{} and confirms it as a suitable tool regarding FAK simulations.\\
\\
In \autoref{multiProt} we investigated the interactions between multiple FAK molecules on a membrane. We interpreted the observations as a tendency to a chain like arrangement of the FAK molecules along their long axis and tried to estimate the impacts of such arrangements on the configurations of the involved FAK molecules. Our results show only a small impact on the quantities associated with FAK activation. Therefore we can not draw the conclusion, that these arrangements could promote activation of FAK in larger clusters.\\
\\
Currently we try to cluster the obtained configurations with more general approaches in order to reveal the parameters inducing configurational changes in multiple FAK interactions. However, arrangement of large molecules such as FAK is a time consuming process, which has not come to an end in our simulations. Longer simulation times could therefore give new insights into the consequences of FAK clustering. 