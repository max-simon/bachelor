\chapter{Conclusion}
The aim of the present work was to gain insights into conformational changes of FAK due to membrane binding and multiple FAK interactions using MD.
\\
In order to avoid falling of FAK on the membrane we introduced a stabilizing force. Although, we showed in \autoref{forceana} that this approach doesn't influence the observables used in the remaining part, it has limitations. Therefore, the cause of the falling is still an important question and should be addressed in further studies e.g. with force distribution analysis methods. Also alternatives to our approach, such as flat-bottom potentials or cylindrical pulling \autocite[p. 156-158]{gromacsManual}, can be considered.\\
\\
From the configurations of FAK in solution we identified important residues contributing to the FERM-kinase interface. The observations fit well with experimental studies of \textcite{structFAK}. Also the burying of the active site of the kinase as well as the hiding of the autophosphorylation site was observed in these simulations.\\
We recognized configurational changes of FAK when it binds to a \pip{} containing membrane, namely a partial opening and the promotion of the autophosphorylation site; the domains stays associated. These changes are consistent with previous studies on the influence of \pip{} binding to FAK by \textcite{pap001} and \textcite{pap003}. Also the free energy profile of the \pip{} binding site in the FERM domain, which we obtained from \martini{} simulations, samples the equivalent profile from \charmm{} simulations in the range of the statistical error.\\
Albeit \martini{} is a coarse graining force field, even atomistic details were reproduced in this thesis, which shows the power of \martini{} and confirms it as a suitable tool regarding FAK simulations.\\
\\
In \autoref{multiProt} we investigated the interactions between multiple FAK molecules on a membrane. First, we examined FERM-FERM dimers and were able to reproduce important characteristics from experimental observations. This indicates that \martini{} is also suitable for the investigation of protein-protein interactions. Afterwards we considered larger clusters. Although size and shape of the clusters differ a lot, we interpreted our observations as a tendency to a chain like arrangement of the FAK molecules along their long axis and tried to estimate the impacts of such arrangements on the configurations of the involved FAK molecules. Our results show only a small impact on the quantities associated with FAK activation. Therefore we can not draw the conclusion, that these arrangements could promote activation of FAK in larger clusters.\\
\\
Currently we try to cluster the obtained configurations with more general approaches in order to reveal the parameters inducing configurational changes in multiple FAK interactions. However, arrangement of large molecules such as FAK is a time consuming process, which has not come to an end in our simulations. Longer simulation times could therefore give new insights into the consequences of FAK clustering. 