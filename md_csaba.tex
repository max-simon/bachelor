\section{Molecular Dynamics simulations}
It is often difficult to get a precise insight into dynamics of a biological system on a molecular scale, since the experimentally feasible resolution in time- and length scales is usually larger than those of the systems and processes of interest.\\
Therefore, most effects can only be measured indirectly (for example, through bulk ensemble methods). Another problem are disturbing factors, which can be hard to control.\\
These problems do not exist in computer simulations, in which the positions of all particles of the system are known to high precision and a much higher time resolution than the dynamics in the system.\\

%TODO %%%%%%%%%%%%%%%%%
% Review changes from here
%TODO %%%%%%%%%%%%%%%%%


%resolution sounds good, time scale could be bad too.
An important tool is Molecular Dynamics simulation (MD). MD can provide atomistic details, but e.g. with coarse-grained models (see \autoref{subsub:coarsegraining}) a length scale of several nanometres is still accessible for microseconds. Therefore, MD is suitable for membranes, proteins and other biomolecules.\\
%add 1-2 sentences to clarify...
In this thesis, MD was used to investigate interactions of FAK with other FAK molecules and the membrane. Therefore, the main concepts of MD and characteristics of the used models are outlined below. However, within the scope of this thesis, it can just be a small survey.\\
Because \gromacs{} (Groningen Machine for Chemical Simulations) \autocites{gromacs1, gromacsManual} was used as MD engine in this thesis, the following explanations refer to \gromacs{} conventions and features.
\subsection{The physics behind MD}
In MD, the system is simulated on an atomistic scale but with classical mechanics only. In order to do so, atoms are only treated as spheres without distinction between electrons and nuclei. Quantum mechanical (QM) effects, such as excitation or electron transfer processes, are therefore not accessible in MD. However, the atoms are parametrized by effective parameters, which can be motivated from QM \autocite[p. 127f]{greenBook}.\\
\\
Newton's equation of motion
\begin{equation}
\vec{F}_i = m_i\frac{\diff^2 \vec{r}_i}{\diff t^2},
\end{equation}
where $\vec{F}_i$ is the force acting on particle $i$, $m_i$ its mass and $\vec{r}_i$ its position, can be turned into two first-order differential equations:
\begin{align}
\label{eq:eq_of_mot_first_order}
\frac{\diff \vec{r}_i}{\diff t} = \vec{v}_i \\
m_i \frac{\diff \vec{v}_i}{\diff t} = \vec{F}_i .
\end{align}
%equations are part of the text, add commas and periods as needed.
They can be integrated numerically, with e.g. Leapfrog- or Verlet integration schemes. Both are low order, i.e. the integration error is $\mathcal{O}(\Delta t^3)$, which saves computational cost. However, they have the great advantage to be time reversible and symplectic. The latter means that it conserves the phase space volume, like the Hamiltonian operator would do as well. Therefore, the long-term error in energy conservation stays small \autocite[p. 72ff]{UnderstandingMD}.\\
\\
The force acting on particle $i$ is given by the potential at its position:
\begin{equation}
\vec{F}_i = - \frac{\partial V}{\partial \vec{r}_i} .
\end{equation}
In MD, bonded and non-bonded interactions contribute to the potential $V$, but it is also possible to apply external forces.
\subsubsection{Bonded interactions}
Bonded interactions act intramolecularly and describe chemical bonds. They can occur between two, three or four particles, which are generally called stretching, bending and twisting (or torsion).\\%I added the descriptions, feel free to cut them.
\\
An interaction between two bonded particles refers to their bond length. A deviation from the equilibrium bond length results in a potential energy, which is usually described by a harmonic oscillator:
\begin{align}
V_{\text{dist, bond }i} = \frac{k_{\text{dist}}}{2}\left( r - r_{0} \right)^2.
\end{align}
$k_{\text{dist,}}$ is the force constant and $r_0$ is the equilibrium bond length for the bond type of bond $i$. For larger deviations, a Morse potential (exponential decaying potential for large deviations) is more precise, but has a much higher computational cost.\\
\\
Also a deviation from an equilibrium angle between three bonded partners results in potential energy. A common description is the harmonic oscillator as well:
\begin{equation}
V_\text{angle} = \frac{k_\text{angle}}{2}\left( \theta - \theta_0 \right)^2 .
\end{equation}%TODO: write what is what or is this clear? no I think it's OK.
\\
The dihedral angle is the angle between two particles separated by three bonds and can therefore be understood as the torsion angle of the intermediate two particles and bond.%CS: I don't know what this means??
Besides a description of torsion, this angle can also be used to preserve plane rings and the chirality of four-particle groups. It is usually approximated with a periodic potential:
\begin{equation}
V_{\text{dihedral, periodic}} = \frac{k_\text{dihedral}}{2}\left(1 + \cos\left(n \phi - \phi_0 \right)\right),
\end{equation}
where $k_\text{dihedral}$ describes the energy barrier for turning the dihedral angle, $n$ the number of minima in the energy function (multiplicity) and $\phi_0$ a phase factor \autocite[p. 71-83]{gromacsManual}.
\subsubsection{Non-bonded interactions}
Non-bonded interactions are present between all atoms in the system and act pairwise. In MD, Pauli repulsion, van der Waals (vdW) forces and electrostatic forces are taken into account. Because bonded interactions use effective parameters, they are usually excluded from non-bonded interactions. %? CS: excluded in what sense?
\\
\\
The Lennard-Jones potential combines Pauli repulsion ($r^{-12}$ term) and the vdW force ($r^{-6}$ term):
\begin{equation}
V_\text{Lennard-Jones} = \sum_{\text{non-bonded pairs i,j}} 4 \epsilon\left(\left(\frac{\sigma}{r_{ij}}\right)^{12} - \left(\frac{\sigma}{r_{ij}}\right)^6\right).
\end{equation}
$\epsilon$ is related to the potential depth and $\sigma$ to the potential range.\\
\\
The Coulomb potential is given by
\begin{equation}
V_\text{Coulomb} = \frac{q_1 q_2}{4 \pi \epsilon_0 \epsilon_r r},
\end{equation}
where $q_1$, $q_2$ are the (partial) charges of the interacting particles, $r$ the distance between them and $\epsilon_r$ the relative dielectric constant \autocite[p. 65-71]{gromacsManual}.\\
\\
In general, non-bonded interactions act between all atoms in the system, which come along with a very large computational cost.\\
The easiest solution is to use a cut-off radius $r_c$. Particles behind this radius are not taken into account. This can be implemented very efficiently with Verlet neighbour lists. For each particle, a neighbour list is created, which contains all particles inside a second radius $r_v$ with $r_v > r_c$. For a force calculation, only the distances to particles on the list have to be calculated. The lists are updated if the maximal displacement in the system is larger than $r_v - r_c$. This method is suitable especially for Lennard-Jones potential, since it decays rapidly and therefore $r_c$ can be chosen very small \autocite[p. 144]{greenBook}.\\
The electrostatic potential is proportional to $1/r$, i.e. has a relatively sluggish decay, and therefore the use of a cut-off radius would lead to large jumps in the potential. Long-range interactions have to be considered, which can be effectively done with Particle Mesh Ewald (PME) summation \autocite{pme}. Particle mesh methods split up the electrostatic potential into a short-range and a long-range part via a switching function. The short-range part can be calculated with a small cut-off radius in real space. The long-range part, however, is calculated by solving the Poisson equation of the actual charge distribution, for which a discrete grid (mesh) is used. In PME, this grid is transformed to Fourier space, where the solution of the Poisson equation is a sum over the gridpoints. This requires of course periodic boundary conditions (see \autoref*{subsec:pbc}) \autocite[p. 246-251]{greenBook}.
\subsubsection{External forces}
In addition to bonded and non-bonded interactions, external or artificial forces can be applied, such as pulling forces, restraints and constraints.\\
\\
With \gromacs{}, it is possible to perform pulling forces onto groups of atoms in the system. In this thesis, pulling was used to bias distances between groups. For this, \gromacs{} provides an option to apply an umbrella potential to two groups, yielding in a force proportional to the deviation of the distance between the groups from a reference distance (as a harmonic potential). The force can be applied on one or two spatial dimensions only or along a predefined vector and the reference distance can change in time  \autocite[p. 154-159]{gromacsManual}.\\
Restraints are artificial potentials applied to positions, distances, angles, dihedral angles or orientations of particle groups
%CS: what do you mean with ``orientations of particle groups'' ?
to conserve labile configurations or to take additional information from experimental data into account \autocite[p. 84f]{gromacsManual}.\\
Constraints are set to keep properties, such as bond lengths, constant. In order to do so, these properties get reset to the desired value after each time step \autocite[p. 44f]{gromacsManual}. % Lincs, shake?
\subsection{Force fields}
%CSABA: this should definitely be in the methods!!!
%And a bunch of stuff from before too, I think!
The parameters for the potentials described above are provided by force fields. There is a wide range of force fields, which are optimized for different application fields. In this thesis, tow different ones, CHARMM36 and Martini, are introduced.\\
\\
Force fields define specific atomtypes, which allow a mapping of atoms (particles in physical system) onto beads (particles used in the simulation). This mapping can take the environment of the particle (binding partners, solvents, nearby charges, etc.) into account, but can also neglect details by e.g. mapping several atoms to one bead (coarse-graining).\\
Force fields not only define the atom types and their properties, but also all parameters for the calculation of the potential (force constants, equilibrium distances, etc.). In short, they define the physics of the system.
\subsubsection{All-atom and the CHARMM36 force field}
CHARMM (\textit{Chemistry at HARvard Macromolecular Mechanics}) is an MD engine, which uses its own force fields. They are bunched together as CHARMM force fields and are provided for other MD engines as well. The number of the force field corresponds to the version number of CHARMM in which the force field was used first. The force field CHARMM36 \autocites{charmm36_protein}{charmm36_lipids} (C36) was published in 2010.\\
\\
In C36, all atoms are considered (all-atom force field). Parameters were mainly optimized to structural experimental data, such as nuclear magnetic resonance (NMR) or X-ray data, but also QM and semi-empirical QM calculations were used (e.g. for dihedral angles of the sidechains of proteins \autocite{charmm36_protein} and partial charges of lipids \autocite{charmm36_lipids}).\\
All simulations for optimizing have been done with a time step of 2$\si{\femto\second}$. Therefore, very detailed dynamics are still included in the simulations.\\
\\
There are several models for water which can be used with all-atom force fields. They differ mainly in intermolecular interactions of water atoms. For simulations in this thesis, the TIPS3P water model was used. It is a modification of the TIP3P water model \autocite{tip3p} used in parametrisation of C36, which allows also Lennard-Jones interactions on the hydrogens and not only on the oxygen \autocite{charmm36_protein}. % TODO: better citing?
\subsubsection{Coarse graining and the Martini 2.2 force field}
\label{subsub:coarsegraining}
The Martini 2.2 \autocites{martini22_lipids}{martini22} force field was introduced in 2013 and is one of the most famous coarse graining force fields.
%better: introduce Martini, no version here (put it in the Methods)
It maps usually four heavy atoms onto a single bead (ring-like structures need a higher resolution), which implies of course a loss of chemical information, but also an enormous reduction of computational cost. With this approach, much larger time- and spatial scales are accessible for MD simulations.\\
\\
The parametrisation in Martini is mainly based on fitting partitioning free energies of small molecules, such as amino acid side chain analogues, between a range of polar and non-polar solvents to results from all-atom simulations and experimental. For lipids, also thermodynamic properties, such as area per lipid, have been considered. The parameters for membrane-protein interactions were optimized by fitting binding energies of small peptides and a membrane to experimental data and results from all-atom simulation data. In the simulations for optimization a time step of 20$\si{\femto\second}$ up to 30$\si{\femto\second}$ was used \autocites{martini22}{martini22_lipids}.\\
\\
Coarse graining brings a lot of side effects along. First of all, the coarse graining is not unique, therefore important structural properties (e.g. the lipid tail length) are neglected.
%the coarse graining is not unique, therefore?? not unique I take to mean that you could have defined different boundaries between beads, but why does that neglect properties? even if there were only 1 coarse graining possible, it would still lose information.
In proteins the simplifications also lead to problems, because the secondary structure becomes less stable in coarse-grained models and has to be constrained by elastic networks (harmonic restraints between backbone beads) \autocite[p. 6812]{martini22_check}.\\
Furthermore, coarse-grained beads have a larger size, which leads to a smoothing of the free energy profile (FEP). Because smaller local minima in the FEP, which would slow down the evolution of the system, are smoothed out, coarse graining speeds up the dynamics in the system. The speedup is not constant, but can be relatively well approximated by a factor of four (obtained in most diffusion simulations) \autocites[p. 6810]{martini22_check}[p. 7815]{martini}.\\
\\
Coarse graining also has an effect on the entropy and the temperature dependency of the system. In NpT ensembles, the Gibbs free energy $G$ is given by
\begin{equation}
G = H - T S
\end{equation}
where $S$ is the entropy and $H$ the enthalpy. Due to a lower number of degrees of freedom in coarse-grained systems, the configurational entropy is reduced. Because Martini is tuned to free energy calculations, this implies also a reduction of the enthalpy. Therefore, a decomposition of $G$ might not be realistic \autocite[p. 6811]{martini22_check}.\\
\\
In Martini, solvent beads represent four water molecules. Unfortunately, the Martini water has a freezing temperature of up to 300 $\si{\kelvin}$ and the freezing process is very sensitive to nucleation. To prevent this antifreeze beads were introduced, which have the same properties as the standard water beads except for a larger $\sigma$ for the Lennard-Jones potential. By changing around 10\% of the standard solvents to these antifreeze beads, the lattice conformation is disturbed and the freezing temperature is increased \autocite[p. 7815]{martini}.\\
The treatment of water as uncharged beads also implies, that the water phase is not polarizable in Martini. This problem is addressed by assuming a uniform relative dielectric constant, but at water phase interfaces the electrostatic interactions are systematically wrong and the interaction strength of polar beads is underestimated. To overcome this a polarizable water (PW) model was introduced to Martini. PW consists of three beads with equal mass. There is one positively charged bead and one negatively charged, both are bound to the central neutral bead. The charged beads acting only electrostatical with other charged beads, but not intramolecular. To control the distribution of the dipole momentum the binding angle between the three beads is constrained. The central bead interacts via the Lennard-Jones potential \autocite{polarizableMartini}. However PW comes with a higher computational cost and is not as well tested as the standard water model.
\subsection{External constraints}
\subsubsection{Periodic boundary conditions}
\label{subsec:pbc}
Because the simulation of an open system is not possible, boundary conditions have to be considered. Closed boundaries often lead to surface interaction artefacts and are therefore in most cases not suitable for MD simulation. That is why usually periodic boundary conditions (PBC) are used.\\
To use PBC the shape of the simulation box has to have a space filling geometry (e.g. rectangular or rhombic dodecahedron). With periodic boundary conditions images of the simulation box are repeated in every direction. If a particle leaves the simulation box, its periodic image is coming in from the opposite. So the number of particles is kept constant while surface interactions are avoided.\\
A particle interacts only with the nearest image of another particle, which means that particles near boundaries can interact with periodic images of other particles instead of the real particle. Nevertheless molecules can have long range interaction with their own periodic images, which leads to artefacts and has to be considered when choosing the systems size.\\
It is generally thought, that PBC only have small or no effects on equilibrium properties or structures of fluids, but known problems are the absence of long wavelength fluctuations (e.g. in phase transitions) or the violation of angular momentum conversation \autocite[p. 141f]{greenBook}.
\subsubsection{Thermostats}
Integration schemes in MD are designed to conserve the energy of a system. However in real biological systems not the energy but rather temperature is kept constant. This can be achieved with thermostats.
\paragraph{Berendsen and Parinello-Bussi}
The Berendsen thermostat \autocite{berendsen} couples the system weakly to a heat bath with temperature $T_0$ by scaling the velocities of the particles. Weak coupling means, that the energy difference is not conducted in one rescaling, but over a given time scale $\tau_T$, which allows a specification of the coupling strength. As the rescaling of velocities transfer kinetic energy from internal degrees of freedom, such as vibrations, to translational and rotational kinetic energy of the center of mass of the system \autocite[p. 738]{velRescaleSucks}, this thermostat is not suitable for production runs. However deviations from $T_0$ decay exponential making this thermostat usefull for equilibration runs or non equilibrium MD simulations \autocite[p. 3689]{berendsen}.\\
The Parrinello-Bussi  thermostat \autocite{parinelloBussi} extends the Berendsen thermostat by a stochastic term, which leads to a more accurate sampling of phase space. The advantages of the Berendsen thermostat are still in place \autocite[p. 31]{gromacsManual}.
\paragraph{Nosé-Hoover thermostat}
The Nosé-Hoover thermostat \autocites{nosehooverthermo}{nosehooverthermo2} extends the Hamiltonian of the system by a friction representing a heat bath, which is represented as a first order derivation in the equations of motion. The friction parameter varies in time and depends on the derivation of the temperature in the system from the given target temperature. For larger deviations Nosè-Hoover thermostat are unstable and lead to large temperature deviations. However the sampling of the phase space of the system is better than in Berendsen thermostats and for that reason Nosè-Hoover can be used for production runs of equilibrated systems \autocite[p. 32f]{gromacsManual}.\\
\\
Often groups are coupled to independent thermostats. This is helpful, because the heat exchange between f.e. proteins and solvents is often not correct. Therefore proteins would cool down and the solvent would heat up \autocite[p. 34]{gromacsManual}.
\subsubsection{Barostats}
In biological systems often not the volume but the pressure is constant, which can be achieved in MD with different barostats.
\paragraph{Berendsen barostat}
Analogously to the Berendsen thermostat the Berendsen barostat \autocite{berendsen} couples the systems pressure weakly to an external pressure $P_0$ by rescaling the positions of the particles. Similar to the Berendsen thermostat this barostat does not sample the whole phase space and should therefore only be used in equilibration runs and non-equilibrium MD simulations \autocite[p. 36]{gromacsManual}.
\paragraph{Parinello-Rahman barostat}
In the Parinello-Rahman barostat \autocites{parinelloBarostat}{parinelloBarostat2} the position vectors of the particles are transformed by a matrix $\underline{M}$. $\underline{M}$ depends on the current pressure, the target pressure and the volume. The Parinello-Rahman barostat samples the full phase space and can be therefore used in production runs. Large deviations from the desired pressure however lead to oscillations in the box, therefore it should not be used for equilibration runs \autocite[p. 36]{gromacsManual}.\\
\\
\gromacs{} provides the possibility to couple the z-direction independently from the x- and y-direction, which is called semiisotropic pressure coupling. This feature is useful for membrane and pulling simulations, because the dynamics differ a lot between these axes.



%For the thermostats used in this thesis some characteristics are outlined below.
%\paragraph{Berendsen thermostat}
%The Berendsen thermostat \autocite{berendsen} couples the system weakly to a heat bath with temperature $T_0$ by scaling the velocities of the particles. The temperature $T$, to which the system is set to, is given by
%\begin{equation}
%\label{eq:berendsen_time_evolving}
%\frac{\diff T}{\diff t} = \frac{T_0 - T}{\tau_T}
%\end{equation}
%Therefore an exponential relaxation can be observed. Weak coupling means, that the energy difference is not conducted in one rescaling, but over a given time scale $\tau_T$, which allows a specification of the coupling strength.\\
%It has been shown, that rescaling of velocities transfer kinetic energy from internal degrees of freedom, such as vibrations, to translational and rotational kinetic energy of the center of mass of the system \autocite[p. 738]{velRescaleSucks}, which results in a wrong sampling of phase space. Therefore this thermostat is not suitable for production runs. Due to the exponential decay it adjusts quit fast, is stable against large deviations from the desired temperature and prevents oscillations. For these reasons it is often used in equilibration runs or non-equilibrium MD simulations \autocite[p. 3689]{berendsen}.
%\paragraph{Parinello-Bussi thermostat}
%The Parrinello-Bussi  thermostat \autocite{parinelloBussi} extends the Berendsen thermostat by a stochastic term, which leads to a more accurate sampling of phase space. The advantages of the Berendsen thermostat are still in place \autocite[p. 31]{gromacsManual}.
%\paragraph{Nosé-Hoover thermostat}
%The Nosé-Hoover thermostat \autocites{nosehooverthermo}{nosehooverthermo2} extends the Hamiltonian of the system by a friction representing a heat bath. The equations of motion are modified to
%\begin{equation}
%\frac{\diff^2 \vec{r}_i}{\diff t^2} = \frac{\vec{F}_i}{m_i} - \frac{p_\xi}{Q(T_0, \tau_T)}\frac{\diff \vec{r}_i}{\diff t}
%\end{equation}
%Here $\xi$ represents the friction parameter with momentum $p_\xi$ and a mass parameter $Q(T_0, \tau_T)$, which defines the coupling strength. $Q$ depends on the target temperature $T_0$ and a coupling time scale $\tau_T$. The evolution of $\xi$ is defined as
%\begin{equation}
%\frac{\diff p_\xi}{\diff t} = T - T_0
%\end{equation}
%This leads to oscillations between the system and the heat bath with the period $\tau_T$. The relaxation is about five times slower as by an exponential relaxation (see Berendsen or Parrinello–Bussi thermostats) with the same $\tau_T$ \autocite[p. 32f]{gromacsManual}.\\
%\\
%Often groups are coupled to independent thermostats. This is helpful, because the heat exchange between f.e. proteins and solvents is often not correct. Therefore proteins would cool down and the solvent would heat up \autocite[p. 34]{gromacsManual}.
%\subsubsection{Barostats}
%In biological systems often not the volume but the pressure is constant. This refers to the isobaric isothermal ensemble (NpT). Below the characteristics of the barostats used in this thesis are outlined.
%\paragraph{Berendsen barostat}
%Analogously to the Berendsen thermostat the Berendsen barostat \autocite{berendsen} couples the systems pressure $P$ weakly to an external pressure $P_0$ by rescaling the positions of the particles.\\
%\begin{equation}
%\label{eq:berendsen_pressure_time_evolving}
%\frac{\diff P}{\diff t} = \frac{P_0 - T}{\tau_P}
%\end{equation}
%$\tau_P$ is the time scale of coupling. Similar to the Berendsen thermostat this barostat does not sample the NpT ensemble and should therefore only be used in equilibration runs and non-equilibrium MD simulations \autocite[p. 36]{gromacsManual}.
%\paragraph{Parinello-Rahman barostat}
%In the Parinello-Rahman barostat \autocites{parinelloBarostat}{parinelloBarostat2} the equations of motion of the particles change to
%\begin{equation}
%\frac{\diff^2 \vec{r}_i}{\diff t^2} = \frac{\vec{F}_i}{m_i} - \underline{M} \frac{\diff \vec{r}_i}{\diff t}
%\end{equation}
%where $\underline{M}$ is a matrix is given by a differential equation depending on the current pressure, the target pressure, the volume and a time scale. The Parinello-Rahman barostat samples the full phase space and can be therefore used in production runs. Large deviations from the desired pressure however lead to oscillations in the box, therefore it should not be used for equilibration runs \autocite[p. 36]{gromacsManual}.\\
%\\
%\gromacs{} provides the possibility to couple the z-direction independently from the x- and y-direction, which is called semiisotropic pressure coupling. This feature is useful for membrane and pulling simulations, because the dynamics differ a lot between these axes.