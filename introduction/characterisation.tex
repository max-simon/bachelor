\section{Characterisation of multiple protein interactions}
For the later analysis, protein-protein interactions are classified using the following terms.
\paragraph{Interaction} Proteins or parts of proteins interact, if their minimal distance is smaller than a cut-off distance.
\paragraph{Neighbour} Protein A is a neighbour of Protein B, if they are interacting. One protein can have several neighbours. For a more detailed characterisation the following types are defined (see also \autoref{tobeadded}):
\begin{enumerate}[label={type \theenumi:}, leftmargin=*]
	\item only the FERM domain interacts with only the FERM domain of the other protein
	\item only the kinase interacts with only the kinase of the other protein
	\item only the FERM domain interacts with only the kinase of the other protein
	\item the FERM domain is interacting with both, the FERM and kinase of the other protein
	\item the kinase is interacting with both, the FERM and kinase of the other protein
	\item the FERM domain is interacting with the FERM domain of the other protein and the kinase is interacting with the kinase of the other protein
	\item the FERM domain is interacting with the kinase of the other protein and the kinase is interacting with the FERM domain of the other protein
\end{enumerate}
\paragraph{Cluster} Protein A belongs to a cluster, if it has at least one neighbour inside the cluster. Two neighbouring proteins form a cluster of size 2. One protein can only belong to one cluster.
\paragraph{Dimer} A dimer is a set of two neighbouring proteins, which do not interact with a third protein.