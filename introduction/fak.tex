\section{Focal adhesion kinase}
% TODO: scaffolding function of FAK, clustering
% TODO: more about cluster shape and so on
Focal adhesions (FA) are macromolecular protein complexes which act as a connection hub between the cell, i.e. the cytoskeleton, and the extracellular matrix (ECM). They enable the cell to performing tension forces, but can also trigger mechanical stimuli from the ECM. One important protein associated to FA is the focal adhesion kinase (FAK). FAK occurs in several signalling pathways and is a key player in integrating extracellular stimuli. It is of large interest not least because in cancer cells often an overexpression of FAK can be found and understanding the activation processes and dynamics of FAK could give rise to new cancer treatments.% \autocite{cancerFAK}.
\subsection{Structure}
FAK consists of four major domains (see \autoref{TOBEADDED}): a FERM domain as N-terminal, a tyrosine kinase, a proline rich region and a focal adhesion targeting (FAT) domain as C-terminal.\\
FERM (4.1 protein, ezrin, radixin and moesin) is a common protein domain, which targets proteins to membranes \autocite{fermdomain} and consists of three subdomains F1, F2 and F3. In the F2 subdomain there is the basic patch  ($^{216}$KAKTLRK$^{222}$), which is a prominent binding site for phosphatidylinositol-4,5-bisphosphate (\pip). This phospholipid is locally generated in FA due to integrin signaling \autocite{pap001}.\\
The kinase domain of the C-lobe, the activation loop and the N-lobe. The catalytic activity of kinase is regulated by the phosphorylation of \acid{Y}{576}\,and \acid{Y}{577}, which are located in the activation loop \autocite{tyrosinePhosphor}. The C-lobe also provides a binding site for \pip\,which is located next to the basic patch of the FERM domain \autocite{pap002}.\\
The FERM domain and the kinase are connected by a linker region. In contrast to other kinase domains the main autophosphorylation site of FAK \acid{Y}{397} can be found in this region \autocite{pap001}.\\ %TODO: better cite
The FAT domain is linked to the kinase by a flexible proline rich region. FAT targets to FA by interacting with talin and paxillin, which are proteins associated with FA \autocite{fatdomain}.
\subsection{Autophosphorylisation and activation}
A maximum catalytic turnover requires \acid{Y}{576}\,and \acid{Y}{577} in the activation loop to be phosphorylated. In the inactive state this region is shielded by the FERM domain. Also the autophosphorylation site \acid{Y}{397} is isolated by the FERM domain. Therefore an activation is only possible if the FERM domain dissociate at least partly from the kinase \autocite{structFAK}. FAK triggers several stimuli, but in this thesis the main focus lies on the allosteric effect of \pip\,binding to the basic patch.\\
% quelle -5: PIP2 and proteins: in- teractions, organization, and information flow
\pip\,is a has a net charge of -4, but in presence of K the deprotonated state gets promoted resulting in a net charge of -5. The electrostatic binding of \pip\,to the basic patch in the F2 subdomain results in long ranged configurational changes, which also influence the interface between the F1 subdomain and the N-lobe. Also the linker region gets less stronger bound so that an autophosphorylisation of \acid{Y}{397} is promoted. Phosphorylated \acid{Y}{397} is a suitable binding site for SH2, a subdomain of proteins from the Src kinase family. The presence of a kinase and the partial opening of the FERM kinase interface leads to a phosphorylisation of \acid{Y}{576}\,and \acid{Y}{577}. The resulting Src-FAK complex can act as an fully active kinase. In this state the FERM domain dissociates from the kinase \autocites{pap003}{pap001}.
\subsection{Dimerization and clustering}
%%%%%%%%%%%%%
% TODO: opening of interface , clustering, characteristics of the clustering
%%%%%%%%%%%%%
The FERM domain induces a dimerization of FAK as it is doing in other proteins containing a FERM domain as well. The interaction emerges around \acid{W}{266} in the connected domains and is stabilised by an interaction of the FAT domain with the basic patch of the other FERM domain respectively \autocite{fakdimers}. It has been shown, that the autophosphorylation happens in trans \autocite{transAuto} and requires \acid{W}{266} \autocite{fakdimers}. On a membrane however the dimer can not be stabilised by the FAT-FERM interaction, because the basic patch binds to \pip\,in the membrane, so the membrane has to stabilize the dimer. Besides dimers larger clusters of FAK can emerge, which can contribute to additional signaling processes \autocite{dimersVsClusters}.
