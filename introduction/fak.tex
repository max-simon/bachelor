\section{Focal adhesion kinase}
Focal adhesions (FA) are macromolecular protein complexes which act as a connection hub between the cell, i.e. the cytoskeleton, and the extracellular matrix (ECM). They enable the cell to performing tension forces, but can also trigger mechanical stimuli from the ECM. One important protein associated to FA is the focal adhesion kinase (FAK). Due to its importance in e.g. cell migration and development \autocite{pap004}, it is of large interest for science.\\% smth about cancer and invasive tumors? \autocite{cancerFAK}.
\\
FAK consists of four major domains (see \autoref{TOBEADDED}): a FERM domain, a tyrosine kinase domain, a proline rich region and a focal adhesion targeting (FAT) domain.\\
FERM (4.1 protein, ezrin, radixin and moesin) is a common protein domain, which targets proteins to membranes \autocite{fermdomain} and consists of three subdomains F1, F2 and F3. In the F2 subdomain there is the basic patch  ($^{216}$KAKTLRK$^{222}$), which is a prominent binding site for phosphatidylinositol-4,5-bisphosphate (\pip), a phospholipid generated locally in FA \autocite{pap001}.\\
The kinase domain consists of the C-lobe, the activation loop and the N-lobe. The activation loop contains \acid{Y}{576}\,and \acid{Y}{577}, which have to be phosphorylated for a maximal activity of the kinase \autocite{tyrosinePhosphor}.\\
FAT is linked to the kinase by the proline rich region. FAT targets to FA by interacting with talin an paxillin, which are proteins associated with FA \autocite{fatdomain}.\\
\\
For an activation of FAK \acid{Y}{576,577} have to be phosphorylated. This is suppressed by the FERM domain, which shields the active loop. Therefore activation of FAK requires an conformational change in the interface of FERM and kinase domain. % describe allosteric, mechanosensing?