\chapter{Setup}
\section{Protein structure}
% TODO: cite gromacs wham?
All simulations have been done with starting configurations adapted from previous work in the group (C36 forcefield: \textcite{pap003}, Martini forcefield: \textcite{sara}). These configurations contain only a FERM-kinase fragment without the FAT domain and its linker (residues 35 to 686, PDB 2J0J \autocite{structFAK}).\\
As explained in \autoref{subsub:coarsegraining} the secondary structure of proteins have to be stabilized in Martini using elastic networks. This was set up by \textcite{sara}. It acts only between residues of the same domain with a force constant of 830 $\si{\kilo\joule\mole^{-1}\nano\meter^{-2}}$. Therefore the interface between FERM domain and kinase is not affected and the linker is still flexible.
\section{Setup 1 - FAK in solution}
\label{setup:setup1}
This setup contains a single FAK molecule in Martini structure, which was placed in a waterbox with ions to eliminate net charges. After a short equilibration the system was simulated for $20\,\si{\micro\second}$ at a temperature of $300\si{\kelvin}$. The standard parameters of the Martini forcefield were used as simulation parameters.
\section{Setup 2 - Free energy of basic patch}
\label{setup:setup2}
For this setup only a part of F2 (residues 107 to 219, referred as F2 lobe in the following), which contains the basic patch, was used. The lobe was placed above a single \pip{} embedded into a phosphatidylethanolamine (\pope{}) membrane (see \autoref{tobeadded}). After a short equilibration the F2 lobe was pulled slowly away from the membrane using a distance pull between the COM of the F2 lobe and the COM of \pip{}. This simulation was used to retrieve starting conformations for the umbrella windows.\\
The starting configuration was set up in C36 and transferred to a Martini (with PW) structure with provided transformation tools \autocite{backwardpy}. In the Martini structures the mentioned elastic network was applied.\\
The number of umbrella windows was chosen accordingly to the sampling (between 90 and 120 windows). Each window was shortly equilibrated and then simulated for $6\,\si{\nano\second}$ in C36 and $10\,\si{\nano\second}$ in Martini (with PW). From the trajectories of the umbrella windows the free energy profile was calculated using \gromacs{} WHAM implementation. For each forcefield five different profiles were obtained to estimate the statistical error.\\
The presented results are based on a total simulation time of $6.33\,\si{\micro\second}$ for Martini, $5.64\,\si{\micro\second}$ for Martini with PW and $3.88\,\si{\micro\second}$ for C36. The temperature in the simulation was $300\si{\kelvin}$. % C36: 1=180, 4.2=135, 5=102, 5.2=102, 6=127, Mart: 1=134, 2=126, 3=130, 4=113, 5=130, MartPol: alle=113
\section{Setup 3 - FAK on a \pip{} membrane}
\label{setup:setup3}
Setup 3 was adopted from \textcite{sara} and contains a single FAK molecule in Martini structure, which was placed on a phosphatidylcholine (\popc{}) and \pip{} (15\%) membrane. Ions were added to eliminate net charges. In contrast to \textcite{sara} the stabilizing force explained in \autoref{motivation} was applied to the protein.\\
Five different copies were simulated for $10\,\si{\micro\second}$ each with a temperature of $300\si{\kelvin}$.
\section{Setup 4 - FAK cluster}
In order to set up a cluster of multiple FAK molecules 25 different frames were chosen from setup 3 equidistant in time. The frames were arranged on a {5x5} grid. The stabilizing force was applied on each FAK molecule independently. After a short equilibration the system was simulated for $9\,\si{\micro\second}$. Five different copies (regarding to the arrangement of the frames) were set up, resulting in a total simulation time of $45\,\si{\micro\second}$. The temperature was set to $300\si{\kelvin}$.